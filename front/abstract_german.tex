%%%%%%%%%%%%%%%%%%%%%%%%%%%%%%%%%%%%%%%%%%%%%%%%%%%%%%%%%%%%%%%%%%%%%%%%%%%%%%%%
% Abstract - German
%%%%%%%%%%%%%%%%%%%%%%%%%%%%%%%%%%%%%%%%%%%%%%%%%%%%%%%%%%%%%%%%%%%%%%%%%%%%%%%%


%--------------------------------------------------------------------------------
\chapter*{Kurzfassung}
\addcontentsline{toc}{chapter}{Kurzfassung}
%--------------------------------------------------------------------------------
Wir leben im großen Datenzeitalter: Immer mehr Datenmengen werden in den letzten Jahren
die von den Anwendern durch Datenerfassung und Simulationen erzeugt werden. Während in großem Maßstab
Analyse und Simulationen haben große Aufmerksamkeit für Cloud Computing erhalten.
und HPC-Systemen, Software zur effizienten Visualisierung großer Datenmengen ist
die kämpfen, um Schritt zu halten.

Wir schlagen vor, Systemsoftware zu erforschen, um große Datenmengen zu vereinfachen und zu beschleunigen.
Visualisierung durch paralleles Rendering, und zur Validierung der Forschung und Entwicklung.
Entwicklung dieser Systemsoftware durch die Entwicklung von neuen Anwendungen für
große Datenvisualisierung.

Diese Forschung und Entwicklung wird es den Wissenschaftlern ermöglichen, die sich mit der Erforschung von Domänen und großen Datenmengen beschäftigen.
Ingenieuren, um die Bedeutung ihrer Daten besser zu extrahieren.
mehr Daten zu erforschen, indem Sie das Rendering beschleunigen und die Verwendung von
hochauflösende Displays, um mehr Details zu sehen.

Aufgrund des Charakters dieser Forschung schlagen wir eine ingenieurgetriebene, iterative
Forschungsprozesses. Basierend auf den Grundlagen eines generischen parallelen Renderings
System können individuelle Forschungsfragen isoliert bearbeitet werden und
optimiert durch datengetriebenes Benchmarking und integriert in die Produktqualität
in das parallele Rendering-System.
