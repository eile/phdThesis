%%%%%%%%%%%%%%%%%%%%%%%%%%%%%%%%%%%%%%%%%%%%%%%%%%%%%%%%%%%%%%%%%%%%%%%%%%%%%%%%
% Abstract - German
%%%%%%%%%%%%%%%%%%%%%%%%%%%%%%%%%%%%%%%%%%%%%%%%%%%%%%%%%%%%%%%%%%%%%%%%%%%%%%%%

\chapter*{Kurzfassung}
\addcontentsline{toc}{chapter}{Kurzfassung}

Daten sind das Gold des 21. Jahrhunderts: Computersimulationen, bildgebende
Verfahren und andere Datenerfassungssysteme generieren immer gr\"ossere
Datenmengen. Visualisierungssoftware zur Darstellung grosser Datenmengen ist,
relativ zu Simulationssoftware und verteilten Systemen f\"ur Cloudumgebungen,
in der Forschung und Entwicklung vernachl\"assigt.

Visualisierung ist ein effizientes Mittel um grosse Datenmengen zu analysieren.
Insbesondere die Visualisierung von dreidimensionalen Datens\"atzen erlaubt ein
intuitives Verst\"andnis der r\"aumlichen Zusammenh\"ange und ihrer Struktur.
Visualisierungshardware steht immer mehr Benutzern zur Verf\"ugung,
insbesondere hochaufl\"osende Monitorw\"ande sind mittlerweile auch f\"ur
kleine Institutionen erschwinglich.

Diese Doktorarbeit besch\"aftigt sich mit paralleler Software und Algorithmen
zur Visualisierung dreidimensionaler Datens\"atze, um diesen Entwicklungen
Folge zu tragen. Als Grundlage f\"ur Forschung und Entwicklung formalisieren
wir die Softwarearchitektur f\"ur paralleles Rendering und stellen unsere
Referenzimplementierung vor. Auf dieser Basis pr\"asentieren wir neue
Forschungsergebnisse und Algorithmen zur schnelleren Visualisierung grosser
Datenmengen. Visualisierungssoftware, welche mit unserer Bibliothek entwickelt
wurde, validiert unseren Ansatz, und erlaubt Benutzern mehr Daten mit besserer
Detail zu analysieren.

