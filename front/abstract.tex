%%%%%%%%%%%%%%%%%%%%%%%%%%%%%%%%%%%%%%%%%%%%%%%%%%%%%%%%%%%%%%%%%%%%%%%%%%%%%%%%
% Abstract - English
%%%%%%%%%%%%%%%%%%%%%%%%%%%%%%%%%%%%%%%%%%%%%%%%%%%%%%%%%%%%%%%%%%%%%%%%%%%%%%%%


%--------------------------------------------------------------------------------
\chapter*{Abstract}
\addcontentsline{toc}{chapter}{Abstract}
%--------------------------------------------------------------------------------
We are living in the big data age: An ever increasing amount of data is being
produced by users through data acquisition and simulations. While large scale
analysis and simulatons have received significant attention for cloud computing
and HPC systems, software to efficiently visualize large amounts of data is
struggling to keep up.

We propose to research system software to facilitate and accelerate large data
visualization through parallel rendering, and to validate the research and
development of this system software by the development of new applications for
large data visualization.

This research and development will enable domain scientists and large data
engineers to better extract meaning from their data, making it feasible to
explore more data by accelerating the rendering and allowing the use of
high-resolution displays to see more detail.

Due to the nature of this research, we propose an engineering-driven, iterative
research process. Based on the foundations of a generic parallel rendering
system, individual research questions can be addressed in isolation and
optimized through data-driven benchmarking, and integrated in product quality
into the parallel rendering system.
